\section{(Co)torsion theory}

The language and methodology follows [BuJaVer]. For the 
following, let $\Cat{A}$ be an abelian category.

\begin{definition}
Let $F: \Cat{A} \to \Cat{A}$ be an endofunctor. We say that
$F$ is a \emph{subobject functor} if there exists a natural 
transformation $\tau$ from the identify functor to $F$
such that for each $X \to Y$, which gives rise to the following:
\[
\begin{diagram}
FX           & \rTo & FY           \\
\dTo{\tau_X} &      & \dTo{\tau_Y} \\
X            & \rTo & Y
\end{diagram}
\]
$\tau_X$ and $\tau_Y$ are injections 
\end{definition}

\begin{definition}
The concept of a quotient functor is dual to that of a suboject
functor. That is, $F: \Cat{A} \to \Cat{A}$ is a \emph{quotient
functor} if there exists a natural transformation $\tau$ from $F$ 
to the identity such that for each $X \to Y$, we have
\[
\begin{diagram}
X            & \rTo & Y            \\
\dTo{\tau_X} &      & \dTo{\tau_Y} \\
FX           & \rTo & FY
\end{diagram}
\]
and $\tau_X$ and $\tau_Y$ are surjections.
\end{definition}

\begin{definition}
We say that $F$ is \emph{idempotent} if $FF = F$.
\end{definition}

\begin{definition}
We say that a subobject functor $F$ is a \emph{pre-radical} (resp.
\emph{pre-coradical}) if for each $X \in \Cat{A}$, $F(X/F(X)) = 0$
(resp. $F(\ker X \to FX) = 0$).
\end{definition}

\begin{definition}
A pre-radical (resp. pre-coradical) $F$ is a \emph{radical} (resp. 
coradical if $F$ is left exact (resp. right exact).
\end{definition}

\begin{rmk}
Notice that being a quotient and being a subobject are dual 
notions. Similarly, (pre-)coradical is just the dual of radical. 
In the following, we consider only subobject functors, and 
(pre-)radical. Dual of the results whose arguments involve only 
``flipping the arrows'' will be stated without proof.
\end{rmk}

\begin{rmk}
In the case where $\Cat{A}$ is small, via Mitchell embedding 
there exists an embedding of $\Cat{A}$ as a subcategory of $R$-mod
for some suitable ring $R$. In this case, the subobject functors, 
quotient functors, idempotence, (pre-)radicals, and (pre-)coradicals
corresponds to their counterparts in the ring theoretic setting.
\end{rmk}

\begin{prop}
Any left exact subobject functor $F$ is idempotent. Dually, any
right exact quotient functor is idempotent.
\end{prop}
\begin{proof}
Fix $X \in \Cat{A}$. Since $F$ is left exact, $F^2X = \ker FX \to 
F(X/FX)$.
However, we have
\[
\begin{diagram}
0    &      & 0       \\
\dTo &      & \dTo    \\
FX   & \rTo & F(X/FX) \\
\dTo &      & \dTo    \\
X    & \rTo & X/FX
\end{diagram}
\]
and since $F(X/FX) \to FX$ is an injection,
\begin{align*}
F^2X &= \ker FX \to F(X/FX) \\
     &= \ker FX \to F(X/FX) \to X/FX \\
     &= \ker FX \to X \to X/FX \\
     &= FX.
\end{align*}
\end{proof}

\begin{prop}
Let $F$ be a preradical of $\Cat{A}$. If $Y$ is a subobject of $FX$,
then $F(X/Y)$ is isomorphic to $FX/Y$. Dually, if $Y$ was a 
quotient of $FX$, then $F(\ker X \to Y)$ is isomorphic to $\ker 
FX \to Y$.
\end{prop}
\begin{proof}
For $0 \to Y \to X \to X/Y \to 0$, consider
\[
\begin{diagram}
0 & \rTo & Y        & \rTo & FX     & \rTo & F(X/Y)   \\
  &      & \dEquals &      & \dInto &      & \dInto   \\
0 & \rTo & Y        & \rTo & X      & \rTo & X/Y      & \rTo & 0
\end{diagram}
\]
Since $F(X/Y) \to X/Y$ is an injection, 
\begin{align*}
\ker FX \to F(X/Y) &= \ker FX \to F(X/Y) \to X/Y \\
                   &= \ker FX \to X \to X/Y \\
                   &= \ker X \to X/Y \\
                   &= Y.
\end{align*}
where the second to last equality comes from the fact that $FX 
\to X$ is an injection.

Applying the Snake Lemma to
\[
\begin{diagram}
0 & \rTo & Y        & \rTo & FX       & \rTo & FX/Y   & \rTo & 0 \\
  &      & \dEquals &      & \dEquals &      & \dTo   \\
0 & \rTo & Y        & \rTo & FX       & \rTo & F(X/Y) 
\end{diagram}
\]
we see that the map $FX/Y \to F(X/Y)$ is an injection. Now
consider
\[
\begin{diagram}
0 & \rTo & FX/Y     & \rTo & F(X/Y) & \rTo & F(X/F(X)) = 0 \\
  &      & \dEquals &        & \dInto &      & \dInto        \\
0 & \rTo & FX/Y     & \rTo   & X/Y    & \rTo & X/F(X)
\end{diagram}
\]
where the bottom sequence is exact, and the top ought to be, but
we do not yet know, exact. Nonetheless, $F(X/Y) \to X/Y \to X/F(X)$
is the 0 map, and therefore, there exists some map $F(X/Y) \to FX/Y$
which is an injection. It follows that $F(X/Y) \simeq FX/Y$.
\end{proof}

\begin{definition}
Let $\Cat{A}$ be an abelian category. A \emph{torsion theory} for
$\Cat{A}$ is a pair $(\Cat{T}, \Cat{F})$ of full subcategories 
called the \emph{torsion subcategory} and \emph{torsion-free 
subcategory} respectively, where the objects of $\Cat{T}$ are
objects $X$ such that $\hom_{\Cat{A}}(X, Y) = 0$ for every $Y
\in \Cat{F}$ and objects of $\Cat{F}$ are objects $Y$ such that
$\hom_{\Cat{A}}(X, Y) = 0$ for every object $X \in \Cat{T}$.
\end{definition}

Certainly $0 \in \Cat{T} \cap \Cat{F}$. Therefore, neither 
subcategories are empty.

\begin{prop}
Let $\Cat{A}$ be a well-powered abelian category with a torsion 
theory, and $\Cat{T}$ and $\Cat{F}$ are two nonempty full 
subcategories. Then, $\Cat{T}$ is a torsion subcategory of 
$\Cat{A}$ if and only if $\Cat{T}$ is closed under extensions, 
direct sums and quotients. Dually, $\Cat{F}$ is a torsionfree 
subcategory of $\Cat{A}$ if and only if $\Cat{F}$ is closed under 
extensions, direct products, and submodules.
\end{prop}

\begin{proof}
It suffices to verify the statement for torsion subcategories.

Suppose $\Cat{T}$ is a torsion subcategory with $\Cat{F}'$ its 
corresponding torsionfree subcategory. 

\emph{Closed under quotients:} suppose $X \in \Cat{T}$. For any 
surjection $X \to Y \to 0$, we have 
\[
0 \to \hom_{\Cat{A}}(Y, F) \to \hom_{\Cat{A}}(X, F)
\]
for any $F \in \Cat{F}'$. However, $\hom_{\Cat{A}}(X, F) = 0$.
Therefore, $\hom_{\Cat{A}}(Y, F) = 0$ for all $F$, and $Y \in
\Cat{T}$.

\emph{Closed under sums:} suppose $\{X_i| i \in I\}$ is a 
collection of objects of $\Cat{T}$. We have
\[
\hom_{\Cat{A}}( \oplus_{i \in I} X_i, F) = \prod_{i \in I}
\hom_{\Cat{A}}( X_i, F) = 0
\]
for all $F \in \Cat{F}'$. It follows that $\oplus_{i \in I} X_i$
is an object of $\Cat{T}$.

\emph{Closed under extensions:} Suppose 
\[
0 \to X' \to X \to X'' \to 0
\]
is an exact sequence in $\Cat{A}$ with $X', X'' \in \Cat{T}$.
Then for any $F \in \Cat{F}$, 
\[
0 \to \hom_{\Cat{A}}(X'', F) \to \hom_{\Cat{A}}(X, F) \to
\hom_{\Cat{A}}(X', F).
\]
Since $\hom_{\Cat{A}}(X'', F) = \hom_{\Cat{A}}(X', F) = 0$,
it follows that $\hom_{\Cat{A}}(X, F) = 0$ for all $F$, and
$X \in \Cat{T}$.

Conversely, suppose $\Cat{T}$ is closed under extensions, direct 
sums and quotients. Let $\Cat{F}'$ be the full subcategory of $Y$ 
such that $\hom_{\Cat{A}}(X, Y) = 0$ for all $X \in \Cat{T}$, and 
$\Cat{T}'$ be the full subcategory of $X'$ such that 
$\hom_{\Cat{A}}(X', Y) = 0$ for all $Y \in \Cat{F}$. We proceed
by showing that $\Cat{T'} = \Cat{T}.$

We show that $\Cat{T} = \Cat{T'}$. Certainly $\Cat{T}$ is a full
subcategory of $\Cat{T'}$. Fix $X \in \Cat{T'}$. Then there exists
a maximal $\Cat{T}$ subobject of $X$. Indeed, let $S = \{X_i| i 
\in I\}$ be the set of subobjects of $X$ in $\Cat{T'}$. Setting 
$X' = \displaystyle \oplus_{i \in I} X_i$, we see that $X''$ is 
an object of $\Cat{T}$, whose image in $X$ is a quotient of 
$X''$, and therefore, is also in $\Cat{T}$. 

Let $\tilde{X}$ be this maximal subobject. We proceed by showing 
that $X/\tilde{X}$ is an object of $\Cat{F}'$, and hence is 0.
Suppose not. Then there exists some $X' \in \Cat{T}$ with a 
nonzero map $f: X' \to X/\tilde{X}$. Since $f(X') \in \Cat{T}$,
replacing $X'$ by its image in $X/\tilde{X}$, we may assume
without loss of generality that $f$ is injective.

Pullback $X \to X/\tilde{X}$ by $f$, and we have:
\[
\begin{diagram}
0         &          & 0           \\
\dTo      &          & \dTo        \\
\tilde{X} & \rEquals & \tilde{X}   \\
\dTo      &          & \dTo        \\
P         & \rTo{i}  & X           \\
\dTo{p}   &          & \dTo        \\
X'        & \rInto   & X/\tilde{X} \\
\dTo      &          & \dTo        \\
0         &          & 0
\end{diagram}
\]
As $i$ is a pullback of an injection, $i$ is itself injective.
Similarly, as $X \to X/\tilde{X}$ is surjective, so is $p$.
Furthermore, $\ker p = \tilde{X}$. Since $\tilde{X}$ and $X'$
are both in $\Cat{T}$, it follows that $P$ must, too. However,
$X'$ is nontrivial, contradicting the maximality of $\tilde{X}$.
Thus, $X/\tilde{X} \in \Cat{F}$, and $X \in \Cat{T}$.
\end{proof}
